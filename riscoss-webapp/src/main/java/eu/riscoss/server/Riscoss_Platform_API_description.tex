\section{Riscoss platform API}

\subsection{Servlets}
\begin{verbatim}
<url-pattern>/api/*</url-pattern>
<servlet-name>RiscossWebApp</servlet-name>
<servlet-class>eu.riscoss.server.ServletWrapper</servlet-class>

<url-pattern>/gauge</url-pattern>
<servlet-name>GaugeServlet</servlet-name>
<servlet-class>eu.riscoss.server.GaugeServlet</servlet-class>

<url-pattern>*.gupld</url-pattern>
<servlet-name>uploadServlet</servlet-name>
<servlet-class>eu.riscoss.server.UploadServiceImpl</servlet-class>

<url-pattern>/models/download</url-pattern>
<servlet-name>downloadServlet</servlet-name>
<servlet-class>eu.riscoss.server.DownloadServlet</servlet-class>
\end{verbatim}

\subsubsection{LayersManager}

\textbf{Path "layers"}
all calls have parameters: 
    @HeaderParam ``token'', 
    @PathParam ``domain''

  GET "/{domain}/list"
    calls db.layerNames()
  POST "{domain}/new" //$deprecated!$
	  @QueryParam("name")
	  @QueryParam("parent")
	  comment: automatic name sanitisation, not notified
	calls db.addLayer( name, parentName );  
  POST "{domain}/create"
  	  @QueryParam("name")
	  @QueryParam("parent")
	  comment: automatic name sanitisation, not notified
	calls db.addLayer( name, parentName );
  DELETE "{domain}/delete"
  	  @QueryParam("name")
      calls db.removeLayer( name );
   GET "{domain}/{layer}/properties/ci"  //contextual info
	  @QueryParam("layer")
    calls db.getLayerData( layer, "ci" );  //reads attributes of a layer node
  PUT "{domain}/{layer}/properties/ci"   //contextual info
	  @QueryParam("layer")
	  @HeaderParam("info")  //json format $move to body!$
	calls db.setLayerData( layer, "ci", json ); //sets attributes of a layer node
  POST "{domain}/edit"  //$deprecated!$
	@QueryParam("name") String name, 
	@QueryParam("newname") String newName )
      calls db.renameLayer( name, newName );
   PUT "{domain}/{layer}/name"
	@PathParam("layer") String name, 
	@QueryParam("value") String newName 
      calls db.renameLayer( name, newName );
      
\subsection{EntityManager}

\textbf{Path "entities"}
all calls have parameters: 
    @HeaderParam ``token'', 
    @PathParam ``domain''
   
   GET "/{domain}/list"  //lists all entities with layer and name
    calls db.entities()    //note: layer retrieved by db.layerOf(name)
    
    GET "/{domain}/list/{layer}"  //lists all entities for a layer
	@PathParam("layer") String layer, 
	calls db.entities(layer)
	
    POST "/{domain}/new"
      @Produces("application/json")
      @QueryParam("name") String name, 
      @QueryParam("layer") String layer, 
      @QueryParam("parent") String parent  $attention single parent...to be removed - extra call for setting parents!$
      comment: automatic name sanitisation, not notified
	calls db.addEntity(name, layer)
	
	POST "/{domain}/create"   //$deprecated!$
	  @HeaderParam("info") String str  $json, split to name (QP), layer (QP), parents (to body!)$
	  comment: for a list of parents
	  
	DELETE "/{domain}/entity/delete"
	  @QueryParam("entity") String entity
	  calls db.removeEntity(entity)
	  
	GET "/{domain}/entity/data"
	  @QueryParam("entity") String entity
	returns layers, parents, children, rdcs, userdata (risk data)

\paragraph{Manage entity hierarchy} 	  
      POST "/{domain}/entity/parent"  //for a single parent, to be called multiple times
	  @QueryParam("entity") String entity
	  calls db.assignEntity(entity, parent);

      GET "/{domain}/entity/parent"  $rename to parentS, it returns a json list!$
	  @QueryParam("entity") String entity
	  
      POST "/{domain}/entity/children"  $reorganise, remove ``entity'' etc...$ 
	  note: for a single parent, to be called multiple times
	  note: same than multiple call to parent with switched params
	  @QueryParam("entity") String entity
	  @HeaderParam("entities") String str  $json, put to body!$
	  calls db.assignEntity(child, entity) foreach child.
	  
	GET "/{domain}/entity/hierarchy"
	  @QueryParam("entity") String entity
	  //returns parents and children in a json list
	
\paragraph{Risk Data Collectors Management} 
  GET "/{domain}/rdcs/list"  //lists risk data collectors enabled for an entity and their parameters
	@QueryParam("entity") String entityName,
	  calls db.isRDCEnabled, db.getRDCParmeter
   PUT "/{domain}/rdcs/save"
      @QueryParam("entity") String entityName
      @HeaderParam("rdcmap") String rdcmapString  ((json format)) $move to body!$
        calls db.setRDCEnabled(entityName, rdcName, enabled); db.setRDCParmeter
    
    GET /{domain}/rdcs/newrun
	  @QueryParam("entity") String entityName
	  //runs rdc.getIndicators for every registered rdc, stores data in RDR	  
	  
	  
\paragraph{RiskData Access} 
	GET "/{domain}/entity/rd/get" $reorganise, remove ``entity'' etc...$
	  @QueryParam("entity") String entity
	  @HeaderParam("entities") String str  $json, put to body!$
	  calls db.readRiskData(entity, id)
	  
\paragraph{Risk Analysis results} 
	GET "/{domain}/entity/rd/get" $reorganise, remove ``entity'' etc...$
	  @QueryParam("entity") String entity
	  calls db.readRASResult(entity);

\subsection{AuthManager}

\textbf{Path "auth"}

      POST "/login"
	@HeaderParam("username") String username, 
	@HeaderParam("password") String password
	logs in on the DB,
	returns the new token
	
	GET ``/token''  $ change to POST?!$
	@HeaderParam("token") String token
	Opens the database with this token.
	returns OK
	
	POST "/register"
	@HeaderParam("username") String username, 
	@HeaderParam("password") String password
	registers AS GUEST! 
	returns OK
	
	GET ``/username''
	@HeaderParam("token") String token
	returns the username for this token
	
	GET "/domains/list"  $deprecated$
	@Context HttpServletRequest req   $uses @Context$!
	
	POST "/domains/selected" $deprecated$
	@Context HttpServletRequest req   $uses @Context$!
	@QueryParam("domain") String domain
	
\subsection{ModelManager}

\textbf{Path "models"}
all calls have parameters: 
    @HeaderParam ``token''
    @PathParam ``domain''
    
    GET "/{domain}/list"
      calls db.getModelList()
      
     GET "/{domain}/model/chunklist" 
     @HeaderParam("models") String models   $json, put to body??$
     calls db.getModelBlob( modelName ), extracts all data to json
     //returns also the type of each object (goal/risk/...) and is used in AHP
	
     GET "/{domain}/model/chunks" 
     @HeaderParam("models") String models  $json, put to body??$
      calls db.getModelBlob( modelName ),
      //used in the whatifanalysis, and in the ModelsModule (for showing the content)
      //returns various info, but not the types.
      
      GET "/{domain}/model/get"
      @QueryParam("name") String name 
      calls db.getModelFilename( name ), db.getModelDescFilename( name )
      returns json properties name, modelfilename, modeldescfilename
      
      GET "/{domain}/model/blob"
      @QueryParam("name") String name 
      calls db.getModelBlob( name )
      returns the model content
     
     DELETE "/{domain}/model/delete"
     @QueryParam("name") String name 
      calls db.removeModel( name )
      
      POST "/{domain}/model/changename"
	@QueryParam("name") String name 
	@QueryParam("newname") String newName 
	calls db.changeModelName(name, newName)
     
\subsection{RDCManager}
   
\textbf{Path "rdcs"} Risk data collectors
	GET "/list"
	returns RDC list from RDCFactory.get().listRDCs(), without database calls
	
\subsection{RDRManager}
   
\textbf{Path "rdr"} Risk data repository
all calls have parameters: 
    @HeaderParam ``token''
    @PathParam ``domain''

	POST "/{domain}/store"
	@HeaderParam("json") String string $list of RiskData, should be in body$
	calls db.storeRiskData( o.toString() );
	
\subsection{RiskConfManager}
   
\textbf{Path "rcs"} Risk configurations
all calls have parameters: 
    @HeaderParam ``token''
    @PathParam ``domain''
	
	GET "/{domain}/list"
	@QueryParam("entity") String entity
	calls db.layerOf( entity ); db.findCandidateRCs( layer );
	
	GET "/{domain}/rc/get"
	@QueryParam("name") String name
	For a risk configuration name, returns json with list of *models* associated, and layers with their models
	
	PUT "/{domain}/rc/put"
	@QueryParam("content") String value $json, contains models and layers$
	Sets models and layers/models for an RC
	
	POST "/{domain}/rc/new"
	@QueryParam("name") String name 
	calls db.createRiskConfiguration( name );
	
	DELETE "/{domain}/rc/delete"
	@QueryParam("name") String name 
	calls db.removeRiskConfiguration( name );
	
	
\subsection{AnalysisManager}
   
\textbf{Path "analysis"}
all calls have parameters: 
    @HeaderParam ``token''	
    @PathParam ``domain''
 calls not detailed, since they will actually not changed.  
   
    @GET @Path( "/{domain}/session/list")
    @POST @Path("/{domain}/session/new")
    @GET @Path("/{domain}/session/{sid}/update-data")
	@GET @Path("/{domain}/session/{sid}/missing-data")
	@PUT @Path("/{domain}/session/{sid}/missing-data")
	@GET @Path("/{domain}/session/{sid}/summary")
	@GET @Path("/{domain}/session/{sid}/results")
	@DELETE @Path("/{domain}/session/{sid}/delete")
	@POST @Path("/{domain}/session/{sid}/newrun")
	@POST @Path("/{domain}/new")
	@POST @Path("/{domain}/whatif")
	@POST @Path("/{domain}/ahp")

\subsection{SecurityManager}
TODO when it is finished.

\subsection{DownloadServlet extends HttpServlet }  
direct GET download of model files and description files
doGet <url-pattern>/models/download</url-pattern>
	String domain = request.getParameter("domain");
	String token = request.getParameter("token");
	String modelName = request.getParameter("name");
	String type = request.getParameter("type");  //"desc" or ``model''
	
\subsection{UploadServlet extends UploadAction }  
Upload through the gupload service
<url-pattern>*.gupld</url-pattern>

all calls have parameters: 
	String domain = request.getParameter("domain");		
	String token = request.getParameter("token");
   
	"modelblob"
	ModelUploader.executeAction
	String name = request.getParameter( "name" );
	returns: ``Success'' or the failure message
	comment: automatic name sanitisation, not notified
	
	"modeldescblob"
	ModelDescUploader.executeAction
	String name = request.getParameter( "name" ); //the file name, used only to propose a filename when downloading
	returns: success message or Exception
	comment: automatic name sanitisation, not notified
	
	"modelupdateblob"
	ModelUpdateUploader.executeAction
	String name = request.getParameter( "name" );
	returns: success message or Exception
	comment: automatic name sanitisation, not notified
	
	getUploadedFile  //not used??
	removeItem  //not used??
	



